\section{Conclusions and future work} \label{sec:conclusions}
% The aim of the visualization work package within the DynAIRx project is to generate visual summaries to assist clinicians with medicine optimisation (SMRs are an opportunity to do so). 
The present work started with reviewing techniques to visualise EHR. Using four mock SMRs the domain problem was characterised and data abstraction was generated to carry out the chart-type design space exploration described in the previous sections. Through exploration of the design space, a list of charts was selected and evaluated. The designs were then implemented in a Python package to generate the same visualizations using 2.1M real patient data from the CPRD. We anticipate that the Python package can be usable for other datasets similar to the CPRD data types used in this study.

The contributions of the present work are based on the three main contributions outlined in the method paper by Sedlmair, Meyer, and Munzner \cite{sedlmair2012design}: problem characterisation, that is, analysis and data abstraction using mock SMRs, the design study to present candidate chart types and a Python package-based implementation. 

% The design study approach of the present work is believed to be reproducible and could be applied to other domain problems.

The present work enables us and other visualization designers looking at generating "integrated" dashboards to use the chart types in this study and the Python package implementation to visualise complete patient records to assist clinicians conducting medication review such as SMRs.